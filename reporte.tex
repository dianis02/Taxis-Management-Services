%%Plantilla para la entrega de trabajos
\documentclass{article}
\usepackage{graphicx} %%Estos son paquetes de imagenes
\usepackage[utf8]{inputenc} %%Paquete para simbolos utf8
\usepackage[T1]{fontenc} %%Paquete para la impresion de simbolos y carácteres
\usepackage{hyperref} %%Paquete para manejar hipervinculos
\usepackage[spanish]{babel} %%paquete para el lenguaje
\usepackage{enumerate}
\usepackage{anysize} %%Margenes
\usepackage{amsmath}
\usepackage{amssymb}
\usepackage[usenames]{color}
\usepackage{float}
\author{Barajas Figueroa José de Jesús}
%%\setlength{\parindent}{1cm} %%Especificar sangrias

\begin{document}
\marginsize{3cm}{2cm}{2cm}{2cm} 
\begin{titlepage} %%Inicio de página de presentación
\begin{center}
    
\begin{figure}[ht]
\begin{center}
\includegraphics[width=1.8in]{./img/unamlogo.png}
\end{center}
\end{figure}

\begin{center}
  {\Large \textbf{
      \vspace*{.5in}   
UNIVERSIDAD NACIONAL AUTONOMA DE MEXICO\\
\vspace*{.3in}
FACULTAD DE CIENCIAS}}\\
\vspace*{.3in}
Proyecto Final\\%%nombre del trabajo
\vspace*{.3in}
Fundamentos de Bases de Datos\\
\vspace*{.3in}
Profesor: Gerardo Aviles Rosas\\
Ayudante: Maria del Pilar Hernández Bastida\\
Ayudante: Eduardo \\
Ayudamte: Eduardo \\
\vspace*{.3in}

Barajas Figueroa José de Jesús\\
Ramírez García Diana Isabel\\
\vspace*{.3in}
2019-1 \\%%Fecha
\begin{figure}[H]
\begin{center}
\includegraphics[width=2.5in]{./img/logo.png}
\end{center}
\end{figure}
\end{center}
\end{center}
\end{titlepage}
\section{Diagrama E/R}
El primer boceto de el diagrama lo hicimos prácticamente considerando las entidades y relaciones que nos proporcionaban 
de manera casi literal, esto para ir tomando consideraciones que de primer mano no son tan aproximadas como el diagrama final 
que queremos, por ahora creo que nos tenemos que concentrar en guardar lo mas posible la integridad referencial.\\ 
Decidimos no utilizar herencia en Cliente porque a pesar de tener 3 tipos diferentes de clientes, no varían en sus atributos por lo
que no es necesario utilizar la herencia.Podíamos utlilizar la herencia en las entidades Chofer y Dueño porque  en este caso
la entidad padre sería chofer porque en el caso de uso se nos especifíca que Dueño tiene los mismos atributos más su rfc, sin emabargo
decidimos no utilizarla y en su luagr crear una relación para especificar si un chofer es dueño con la finalidad de facilitar su conversión
al modelo Relacional.\\
\begin{figure}[H]
\begin{center}
\includegraphics[width=5in]{./img/R_boceto1.jpeg}
\caption{Primer revisión 12-noviembre-2018}
\end{center}
\end{figure}
Aqui podemos observar las primeras modificaciones echas al diagrama, en primera instancia, 
convertimos la relacion transportar en una relacion ternaria, con cliente, viaje y chofer, 
ya que anteriormente teniamos viaje y clientes separados, no teniamos una forma de crear un vínculo entre el viaje y el cliente,
a su vez, dejamos intactas las relaciones, perteneces, cubrir, manejar y cometer. 
Sin embargo aun tenemos un poco de duas a cerca de como se quiera manejar el descuento de cada viaje.\\
\begin{figure}[H]
\begin{center}
\includegraphics[width=5in]{./img/R_boceto2.jpeg}
\caption{Segunda revisión 19-Noviembre-2018}
\end{center}
\end{figure}
\begin{figure}[H]
\begin{center}
\includegraphics[width=5in]{./img/R_boceto3.jpeg}
\caption{Diagrama E/R final. 21-Noviembre-2018 }
\end{center}
\end{figure}
En esta nueva versión del diagrama, cambiamos todos los nombres de atributos que tenían símbolos especiales como \# o ?,
definimos también la relación débil,  quitamos unos atributos, por último decidimos tomar a descuento como un atributo calculado.\\
\begin{figure}[H]
\begin{center}
\includegraphics[width=5in]{./img/R_boceto4.png}
\caption{Diagrama E/R final. 26-Noviembre-2018 }
\end{center}
\end{figure}
En esta última versión del diagrama , cambiamos la relación transportar por una relación de grado 4 , al relacionar de igual manera 
a la entidad vehículo porque deseamos saber que coche se utilizó en el viaje,debido a que un Chofer puede manejar distintos autos y 
anteriormente no nos era posible tener esta información con las relaciones que ya teníamos.
Como consecuencia de esto quitamos la relación manejar ya que con transportar podemos conoces que vehículo y cuantos maneja un chofer.
De igual manera decidimos que Viaje fuese una entidad fuerte y de igual manera la relación transportar.Otro cambió que realizamos fue 
cambiar la cardinalidad de asegurar y la volvimos ternaria en la cual agregamos a la entidad Dueño.
\section {Diagrama Relacional}
Una vez completado el diagrama entidad relación, se comenzó con la transformación sintáctica a un diagrama Relacional, 
se aprecia el como se desaparecen los atributos calculados(datos que podemos obtener en tiempo constante sin necesidad de ocupar espacio en la base de datos).
Como es el caso de descuento, ganancia y costo, y nuestro atributos multivaluados generan nuevas tablas como lo son origen , destino y rol.\\
\begin{figure}[H]
\begin{center}
\includegraphics[width=5in]{./img/relacional.jpg}
\caption{Diagrama relacional 21-Noviembre-2018}
\end{center}
\end{figure}


\section{Normalización}
En esta sección vamos a realizar el proceso de normalización mediante la 3FN.\\
\begin{itemize}


\item Persona\\
Sea Persona(id\_persona,nombre,paterno,materno,correo\_electronico,fecha\_ingreso,fotografia\\,delegación,colonia,calle,lote,celular,es\_dueño,es\_chofer,es\_cliente).\\
Para facilitar la normalización renombraremos los atributos como sigue:
id\_persona = A,nombre = B,paterno = C,materno = D,correo\_electronico = E,fecha\_ingreso = F,fotografia = G,delegación = H,colonia = I,calle = J,lote = K,celular = L,es\_dueño = M, es\_chofer = N y es\_cliente = O\\
Sea Persona(A,B,C,D,E,F,G,H,I,J,K,L,M) con las dependencias funcionales 
Infracción = \{A $\rightarrow$ BCDEFGLM, H $\rightarrow$ IJK  \} \\
\\
Buscamos atributos superfluos iquierdos:\\
No tenemos dependencias para buscar elementos superfluos izquierdos.\\
Buscamos atributos superfluos derechos:\\
\\
Sea A $\rightarrow$ BCDEFGLMNO\\
¿C es superfluo? $\Rightarrow$ A $\rightarrow$ BDEFGLMNO \\
\{A\}+= \{A,B,D,E,F,G,L,M,N,O\} No aparece C, entonces no es superfluo.\\
¿B es superfluo? $\Rightarrow$ A $\rightarrow$ CDEFGLMNO \\
\{A\}+= \{A,C,D,E,F,G,L,M,N,O\} No aparece B, entonces no es superfluo.\\
¿D es superfluo? $\Rightarrow$ A $\rightarrow$ BCEFGLMNO \\
\{A\}+= \{A,B,C,E,F,G,L,M,N,O\}  No aparece D, entonces no es superfluo.\\
¿E es superfluo? $\Rightarrow$ A $\rightarrow$ BCDFGLMNO \\
\{A\}+= \{A,B,C,D,F,G,L,M,N,O\}  No aparece E, entonces no es superfluo.\\
¿F es superfluo? $\Rightarrow$ A $\rightarrow$ BCDEGLMNO \\
\{A\}+= \{A,B,C,D,E,G,L,M,N,O\}  No aparece F, entonces no es superfluo.\\
¿G es superfluo? $\Rightarrow$ A $\rightarrow$ BCDEFLMNO \\
\{A\}+= \{A,B,C,D,E,F,L,M,N,O\}  No aparece G, entonces no es superfluo.\\
¿L es superfluo? $\Rightarrow$ A $\rightarrow$ BCDEFGMNO \\
\{A\}+= \{A,B,C,D,E,F,G,M,N,O\}  No aparece L, entonces no es superfluo.\\
¿M es superfluo? $\Rightarrow$ A $\rightarrow$ BCDEFGLNO \\
\{A\}+= \{A,B,C,D,E,F,G,L,N,O\}  No aparece M, entonces no es superfluo.\\
¿N es superfluo? $\Rightarrow$ A $\rightarrow$ BCDEFGLMO \\
\{A\}+= \{A,B,C,D,E,F,G,L,M,O\}  No aparece N, entonces no es superfluo.\\
¿O es superfluo? $\Rightarrow$ A $\rightarrow$ BCDEFGLMN \\
\{A\}+= \{A,B,C,D,E,F,G,L,M,N\}  No aparece O, entonces no es superfluo.\\
\\

Sea H $\rightarrow$ IJK 
¿I es superfluo? $\Rightarrow$ H $\rightarrow$ JK \\
\{H\}+= \{H,J,K\} No aparece I, entonces no es superfluo.\\
¿J es superfluo? $\Rightarrow$ H $\rightarrow$ IK \\
\{H\}+= \{H,J,K\} No aparece J, entonces no es superfluo.\\
¿K es superfluo? $\Rightarrow$ H $\rightarrow$ IJ \\
\{H\}+= \{H,I,J\} No aparece K, entonces no es superfluo.\\
\\
Sea Fmin = \{A $\rightarrow$ BCDEFGLM, H $\rightarrow$ IJK  \}\\
Tendremos las siguientes relaciones resultantes:
\begin{itemize}
\item Persona(id\_persona,nombre,paterno,materno,correo\_electronico,fecha\_ingreso,fotografia,\\ celular,es\_dueño,es\_chofer,es\_cliente)
\item Dirección(delegación,colonia,calle,lote)
\end{itemize}






\item 
Infracción\\ Sea Infracción (num\_infracción,licencia,costo,delegación,colonia,calle,lote).\\
Para facilitar la normalización renombraremos los atributos como sigue:
num\_infracción = A, licencia = B, costo = C, delegación = D, colonia = E , calle = F y lote = G.\\
Sea Infracción(A,B,C,D,E,F) con las dependencias funcionales 
Infracción = \{A $\rightarrow$ CB, D $\rightarrow$ EF \} \\
\\
Buscamos atributos superfluos iquierdos:\\
No tenemos dependencias para buscar elementos superfluos izquierdos.\\
\\
Buscamos atributos superfluos derechos:\\
Sea A $\rightarrow$ CB \\
¿C es superfluo? $\Rightarrow$ A $\rightarrow$ B \\
\{A\}+= \{A,B\} No aparece C, entonces no es superfluo.\\
¿B es superfluo? $\Rightarrow$ A $\rightarrow$ C \\
\{A\}+= \{A,C\} No aparece B, entonces no es superfluo.\\
\\
Sea D $\rightarrow$ EFG\\
¿E es superfluo? $\Rightarrow$ D $\rightarrow$ F \\
\{D\}+= \{D,F,G\} No aparece E, entonces no es superfluo.\\
¿F es superfluo? $\Rightarrow$ D $\rightarrow$ E \\
\{D\}+= \{D,E,G\} No aparece F, entonces no es superfluo.\\
¿F es superfluo? $\Rightarrow$ D $\rightarrow$ E \\
\{D\}+= \{D,E,F\} No aparece G, entonces no es superfluo.\\
\\
Sea Fmin = \{A $\rightarrow$ CB, D $\rightarrow$ EF\}\\
Tendremos las siguientes relaciones resultantes:
\begin{itemize}
\item Infracción(num\_infraccion,licencia,costo)
\item Dirección(delegación,colonia,calle,lote)
\end{itemize}

\item Chofer\\
Sea Chofer(id\_persona,num\_licencia)
Para facilitar la normalización renombraremos los atributos como sigue:
 num\_licencia = A y id\_persona= B 
Sea Chofer(A,B) con las dependencias funcionales 
Ser = \{A $\rightarrow$ B\} \\
\\
Tenemos una única dependencia funcional trivial, ya esta normalizada.
Tendremos las siguientes relaciones resultantes:
\begin{itemize}
\item Chofer(id\_persona,num\_licencia)
\end{itemize}


\item Dueño\\
Sea Dueño(rfc,id\_persona)\\
Para facilitar la normalización renombraremos los atributos como sigue:
rfc = A y id\_persona = B.\\
Sea Dueño(A,B) con las dependencias funcionales 
Dueño = \{A $\rightarrow$ B\} \\
\\
Tenemos una única dependencia funcional trivial, ya esta normalizada.
Tendremos las siguientes relaciones resultantes:
\begin{itemize}
\item Dueño(rfc,id\_persona)
\end{itemize}


\item Vehículo\\
Sea Vehículo(número\_económico,año,modelo,marca,cilindros,descripción\_baja,activo
,num\_puertas, num\_pasajeros, transmición,llanta\_refacción,combustible,fecha\_baja)\\
Para facilitar la normalización renombraremos los atributos como sigue:
número\_económico = A,año,modelo = B,marca,cilindros = C,descripción\_baja = D,activo = E
,num\_puertas = F, num\_pasajeros = G, transmición = H,llanta\_refacción = I,combustible = J,fecha\_baja = K
Sea Vehículo(A,B,C,D,E,F,G,H,I,J,K) con las dependencias funcionales
Vehículo(A $\rightarrow$ BCDEFGHIJK) \\
\\
Tenemos una única dependencia funcional trivial, ya esta normalizada.
Tendremos las siguientes relaciones resultantes:
\begin{itemize}
\item Vehículo(A $\rightarrow$ BCDEFGHIJK) \\
\end{itemize}



\item Aseguradora\\
Sea Aseguradora(rfc,nombre)\\
Para facilitar la normalización renombraremos los atributos como sigue:
rfc = A y nombre = B.\\
Sea Aseguradora(A,B) con las dependencias funcionales 
Aseguradora = \{A $\rightarrow$ B\} \\
\\
Tenemos una única dependencia funcional trivial, ya esta normalizada.
Tendremos las siguientes relaciones resultantes:
\begin{itemize}
\item Aseguradora(rfc,nombre)\\
\end{itemize}


\item Asegurar\\
Sea Asegurar(dueño\_rfc,aseguradora\_rfc, vehiculo\_numero\_económico, num\_seguro,cobertura)\\
Para facilitar la normalización renombraremos los atributos como sigue:
dueño\_rfc = A,aseguradora\_rfc = B, vehiculo\_numero\_económico = C, num\_seguro = D,cobertura = E\\
Sea Asegurar(A,B,C,D,E) con las dependencias funcionales 
Asegurar = \{ABC $\rightarrow$ DE\} \\
\\
Tenemos una única dependencia funcional, entonce no hay otr manera 
de alcanzar los atributos, es decir ya esta normalizada.
Tendremos las siguientes relaciones resultantes:
\begin{itemize}
\item Asegurar = \{ABC $\rightarrow$ DE\} \\
\end{itemize}


\item Viaje\\
Sea Viaje(id\_viaje,fecha,tiempo,distancia,personas)\\
Para facilitar la normalización renombraremos los atributos como sigue:
id\_viaje = A,fecha = B,tiempo = C,distancia = D,personas = E.\\
Sea Viaje(A,B,C,D,E) con las dependencias funcionales
Viaje = \{A $\rightarrow$ BCDE\} \\
\\
Tenemos una única dependencia funcional, entonce no hay otr manera 
de alcanzar los atributos, es decir ya esta normalizada.
Tendremos las siguientes relaciones resultantes:
\begin{itemize}
\item Viaje = \{A $\rightarrow$ BCDE\} \\
\end{itemize}



\item Destino\\
Sea Destino(id\_viaje,delegación,colonia,calle,lote)\\
Para facilitar la normalización renombraremos los atributos como sigue:
id\_viaje = A ,delegación = B,colonia = C,calle = D,lote = E\\
Sea Destino(A,B) con las dependencias funcionales
Destino = \{A $\twoheadrightarrow$ BCDE\} \\
\\
Tenemos una única dependencia funcional trivial, ya esta normalizada.
Tendremos las siguientes relaciones resultantes:
\begin{itemize}
\item Destino(id\_viaje,delegación,colonia,calle,lote))\\
\end{itemize}

\item Origen\\
Sea Origen(id\_viaje,delegación,colonia,calle,lote)\\
Para facilitar la normalización renombraremos los atributos como sigue:
id\_viaje = A ,delegación = B,colonia = C,calle = D,lote = E\\
Sea Destino(A,B) con las dependencias funcionales
Destino = \{A $\twoheadrightarrow$ BCDE\} \\
\\
Tenemos una única dependencia funcional trivial, ya esta normalizada.
Tendremos las siguientes relaciones resultantes:
\begin{itemize}
\item Origen(id\_viaje,delegación,colonia,calle,lote))\\
\end{itemize}

\item Cliente\\
Sea Cliente(id\_cliente,id\_persona,rol,ubicación\_cu)\\
Para facilitar la normalización renombraremos los atributos como sigue:
id\_cliente = A,id\_persona = B,rol = C,ubicación\_cu = D)
Sea Cliente(A,B,C,D) con las dependencias funcionales
Cliente = \{A $\rightarrow$ BCD \} \\
\\
Tenemos una única dependencia funcional, entonce no hay otr manera 
de alcanzar los atributos, es decir ya esta normalizada.
Tendremos las siguientes relaciones resultantes:
\begin{itemize}
\item Cliente(id\_cliente,id\_cliente,rol,ubicación\_cu)
\end{itemize}


\item Rol\\
Sea Rol(cliente\_id, rol)\\
Para facilitar la normalización renombraremos los atributos como sigue:
cliente\_id = A y rol = B.\\
Sea Rol(A,B) con las dependencias funcionales
Rol = \{A $\twoheadrightarrow$ B\} \\
\\
Tenemos una única dependencia funcional trivial, ya esta normalizada.
Tendremos las siguientes relaciones resultantes:
\begin{itemize}
\item Sea Rol(cliente\_id, rol)\\
\end{itemize}

\item Transportar\\
Sea Transportar(id\_viaje,id\_cliente,licencia,vehiculo\_numero\_economico)\\
Para facilitar la normalización renombraremos los atributos como sigue:
id\_viaje = A,id\_cliente = B,licencia = C,vehiculo\_numero\_economico = D.\\
Sea Transportar(A,B,C,D) con las dependencias funcionales
Transportar = \{ABCD $\rightarrow$ ABCD \} \\
\\
Tenemos una única dependencia funcional con todos los atributos. Tenemos
una dependencia tricial, entonces ya esta normalizada, nos quedamos con la
misma relación.
\begin{itemize}
\item Sea Transportar(id\_viaje,id\_cliente,licencia,vehiculo\_numero\_economico) \\
\end{itemize}
\end{itemize}

Después de normalizar tendremos las siguientes relaciones:
\begin{itemize}
\item Infracción\\
\item Chofer\\
\item Persona\\
\item Dueño\\
\item Vehículo\\
\item Asegurar\\
\item Aseguradora\\
\item Viaje\\
\item Destino\\
\item Origen\\
\item Cliente\\
\item Rol\\
\item Transportar\\
\item Dirección\\
\item Dirección_infraccion\\
\\
\end{itemize}

\section{Consultas}
A continuación se presentan algunas consultas que se pueden realizar en la base de datos.
\begin{enumerate}
    \item El nombre de los choferes que hicieron viajes hacia la Facultad de Ciencias durante el mes de Octubre de 2018.
        
    \item Viajes ordenados por delegación y número de veces que se ha realizado un viaje con destino a dicha delegación.
    
    \item El número de viajes que ha realizado cada vehículo.
    
    \item El nombre del personal (alumnos, académicos y trabajadores) que han realizado más viajes a la Facultad de Ingeniería.
    
    \item El promedio de distancias recorridas en viajes realizados durante el semestre 2019-1 (Agosto 2018-Diciembre 2018).
    
    \item Nombre de los choferes que no son dueños.
    
    \item Todas las infracciones durante el mes de diciembre de 2018 con costo mayor a 1500.
    
    \item Las facultades con el numero de viajes como destino.
    
    \item Todos los clientes que son alumnos.
    
    \item Los viajes regristrados el mes de enero de 2018 con destino a la delegación Iztapalapa.
    
    \item El número de infracciones por chofer.
    
    \item Los choferes que nunca han realizado viajes con destino a la Facultad de Medicina.
    
    \item El viaje de mayor distancia recorrida con origen en la delegación Tlahuac realizado por un carro de 4 puertas con destino a la delegación Xochimilco.
    
    \item El número de clientes que son alumnos o académicos pero no trabajadores registrados en el sistema.
    
    \item Todos los vehículos de 4 puertas del año 2012 que han transportado a algún trabajador.
    
    \item El número promedio de personas que reaizaron viajes en el año 2017 y 2018.
    
    \item Todos los choferes que vivien en la delegación Tlahuac y no son dueños.
    
    \item Las aseguradoras que tienen en cobertura vehículos únicamente de 4 puertas.
    
    \item Los vehículos de 4 pasajeros que están asegurados y han realizado viajes hacia la delegación Tlalpan.
    
    \item Todos los vehículos asegurados con cobertura amplia que son del año 2012.
    
\end{enumerate}

\begin{figure}[h]
            \centering
            \includegraphics[width=0.9\textwidth]{img/53.png}
            \caption{Consulta 1}
            \label{fig:my_label1}
        \end{figure}
        
\begin{figure}[h]
            \centering
            \includegraphics[width=0.9\textwidth]{img/54.png}
            \caption{Consulta 2}
            \label{fig:my_label2}
        \end{figure}

\begin{figure}[h]
            \centering
            \includegraphics[width=0.9\textwidth]{img/55.png}
            \caption{Consulta 3}
            \label{fig:my_label3}
        \end{figure}

\begin{figure}[h]
            \centering
            \includegraphics[width=0.9\textwidth]{img/56.png}
            \caption{Consulta 4}
            \label{fig:my_label4}
        \end{figure}

\begin{figure}[h]
            \centering
            \includegraphics[width=0.9\textwidth]{img/57.png}
            \caption{Consulta 5}
            \label{fig:my_label5}
        \end{figure}
        
\begin{figure}[h]
            \centering
            \includegraphics[width=0.9\textwidth]{img/58.png}
            \caption{Consulta 6}
            \label{fig:my_label6}
        \end{figure}
        
\begin{figure}[h]
            \centering
            \includegraphics[width=0.9\textwidth]{img/59.png}
            \caption{Consulta 7}
            \label{fig:my_label7}
        \end{figure}
        
\begin{figure}[h]
            \centering
            \includegraphics[width=0.9\textwidth]{img/60.png}
            \caption{Consulta 8}
            \label{fig:my_label8}
        \end{figure}
        
\begin{figure}[h]
            \centering
            \includegraphics[width=0.9\textwidth]{img/61.png}
            \caption{Consulta 9}
            \label{fig:my_label9}
        \end{figure}
        
\begin{figure}[h]
            \centering
            \includegraphics[width=0.9\textwidth]{img/62.png}
            \caption{Consulta 10}
            \label{fig:my_label10}
        \end{figure}
        
\begin{figure}[h]
            \centering
            \includegraphics[width=0.9\textwidth]{img/63.png}
            \caption{Consulta 11}
            \label{fig:my_label11}
        \end{figure}
        
\begin{figure}[h]
            \centering
            \includegraphics[width=0.9\textwidth]{img/64.png}
            \caption{Consulta 12}
            \label{fig:my_label12}
        \end{figure}
        
\begin{figure}[h]
            \centering
            \includegraphics[width=0.9\textwidth]{img/65.png}
            \caption{Consulta 13}
            \label{fig:my_label13}
        \end{figure}
        
\begin{figure}[h]
            \centering
            \includegraphics[width=0.9\textwidth]{img/66.png}
            \caption{Consulta 14}
            \label{fig:my_label14}
        \end{figure}
        
\begin{figure}[h]
            \centering
            \includegraphics[width=0.9\textwidth]{img/67.png}
            \caption{Consulta 15}
            \label{fig:my_label15}
        \end{figure}
        
\begin{figure}[h]
            \centering
            \includegraphics[width=0.9\textwidth]{img/68.png}
            \caption{Consulta 16}
            \label{fig:my_label16}
        \end{figure}
        
\begin{figure}[h]
            \centering
            \includegraphics[width=0.9\textwidth]{img/69.png}
            \caption{Consulta 17}
            \label{fig:my_label17}
        \end{figure}
        
\begin{figure}[h]
            \centering
            \includegraphics[width=0.9\textwidth]{img/70.png}
            \caption{Consulta 18}
            \label{fig:my_label18}
        \end{figure}
        
\begin{figure}[h]
            \centering
            \includegraphics[width=0.9\textwidth]{img/71.png}
            \caption{Consulta 19}
            \label{fig:my_label19}
        \end{figure}
        
\begin{figure}[h]
            \centering
            \includegraphics[width=0.9\textwidth]{img/72.png}
            \caption{Consulta 20}
            \label{fig:my_label20}
        \end{figure}
\section{Referencias}
\end{document}

